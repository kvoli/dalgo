\documentclass[12pt]{article}
\usepackage{comment}
\usepackage[utf8]{inputenc}
\usepackage{xspace}
\usepackage{gastex}
\usepackage{amsmath}
\usepackage{amssymb}
\usepackage{wrapfig}
\usepackage{tikz}
\usepackage{float}
\usepackage{pgfplots}
\usepackage{booktabs} % For \toprule, \midrule and \bottomrule
\usepackage{pgfplotstable} % Generates table from .csvi
\usepackage{csquotes}
\usepackage{graphicx}
\usepackage{amsmath}
\usepackage{multicol}
\usepackage{tgtermes} % times font
\usepackage[shortlabels]{enumitem}
\usepackage{parskip}

\pagestyle{empty}
\textwidth      165mm
\textheight     252mm
\topmargin      -18mm
\oddsidemargin  -2mm
\evensidemargin -2mm
% \renewcommand{\baselinestretch}{0.96}
\newcommand{\impl}{\mathbin{\Rightarrow}}
\newcommand{\biim}{\mathbin{\Leftrightarrow}}
\newcommand{\id}[1]{\mbox{\textit{#1}}}
\newcommand{\tuple}[1]{\langle #1 \rangle}
\newcommand{\ma}{\mathsf{a}}
\newcommand{\mb}{\mathsf{b}}
\newcommand{\mc}{\mathsf{c}}
\newcommand{\md}{\mathsf{d}}
\newcommand{\nat}{\mathbb{N}}
\newcommand{\intg}{\mathbb{Z}}

\newcounter{question}
\newcommand{\question}[1]{
    \stepcounter{question}
    \thequestion. #1 \hfill
}

\newcommand{\revision}[1]{
    \stepcounter{question}
    \thequestion. #1* \hfill
}



\begin{document}
\topskip0pt
\begin{center}
    {\sc The University of Melbourne
        \\
        School of Computing and Information Systems
        \\
    COMP90020 Distributed Algorithms}
    \bigskip \\
    {\Large\bf Tutorial Week 6: Leader Election}
    \bigskip \\
\end{center}

\section*{Notes}



\section*{Exercises}

\setcounter{question}{24}

\question{In the Bully algorithm, a recovering process starts an election and will become the new coordinator if it has a higher identifier than the current incumbent. Is this a necessary feature of the algorithm?}

\question{Suggest how to adapt the Bully algorithm to deal with temporary network partitions (slow communication) and slow processes.}

\question{Assume that processes do not have a unique identifier (as previously assumed) and are instead anonymous. Is there a deterministic election algorithm for leader election in a ring of $n>1$ processes? Provide a counter-example or provide such an algorithm..}

\question{The problem of leader election has some similarities with the mutual exclusion problem. Last week we discussed Maekawa’s distributed mutual exclusion algorithm with $O(\sqrt{n})$ message complexity. Can we use similar ideas to design a leader election algorithm with sub-linear message complexity?}

\question{In a hypercube of n nodes, suggest an algorithm for leader election with a message complexity of $O(n \cdot \log n)$.}

\end{document}
