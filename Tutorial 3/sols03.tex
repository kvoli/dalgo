
\documentclass[12pt]{article}
\usepackage{comment}
\usepackage[utf8]{inputenc}
\usepackage{xspace}
\usepackage{gastex}
\usepackage{amsmath}
\usepackage{amssymb}
\usepackage{wrapfig}
\usepackage{tikz}
\usepackage{float}
\usepackage{pgfplots}
\usepackage{booktabs} % For \toprule, \midrule and \bottomrule
\usepackage{pgfplotstable} % Generates table from .csvi
\usepackage{csquotes}
\usepackage{graphicx}
\usepackage{amsmath}
\usepackage{multicol}
\usepackage{tgtermes} % times font
\usepackage[shortlabels]{enumitem}
\usepackage{parskip}
\usepackage{import}
\usepackage{pdfpages}
\usepackage{transparent}
\usepackage{xcolor}

\newcommand{\incfig}[2][1]{%
    \def\svgwidth{#1\columnwidth}
    \import{./figures/}{#2.pdf_tex}
}

\pdfsuppresswarningpagegroup=1
\pagestyle{empty}
\textwidth      165mm
\textheight     252mm
\topmargin      -18mm
\oddsidemargin  -2mm
\evensidemargin -2mm
% \renewcommand{\baselinestretch}{0.96}
\newcommand{\impl}{\mathbin{\Rightarrow}}
\newcommand{\biim}{\mathbin{\Leftrightarrow}}
\newcommand{\id}[1]{\mbox{\textit{#1}}}
\newcommand{\tuple}[1]{\langle #1 \rangle}
\newcommand{\ma}{\mathsf{a}}
\newcommand{\mb}{\mathsf{b}}
\newcommand{\mc}{\mathsf{c}}
\newcommand{\md}{\mathsf{d}}
\newcommand{\nat}{\mathbb{N}}
\newcommand{\intg}{\mathbb{Z}}

\newcounter{question}
\newcommand{\question}[1]{
    \stepcounter{question}
    \thequestion. #1 \hfill
}

\newcommand{\revision}[1]{
    \stepcounter{question}
    \thequestion. #1* \hfill
}


\begin{document}
\topskip0pt
\begin{center}
    {\sc The University of Melbourne
        \\
        School of Computing and Information Systems
        \\
    COMP90020 Distributed Algorithms}
    \bigskip \\
    {\Large\bf Solutions Week 4: Global States}
\end{center}

\section*{Exercises}

\setcounter{question}{14}

\question{

    \begin{enumerate}[(a)]
        \item Please see the excalidraw for a visual solution.
        \item Please see the excalidraw for a visual solution.
    \end{enumerate}
}


\question{In answering this question, it's important to make a note that the message being passed around is like a token
    and has no value in it's actual contents. We can see that Q, will always have a state $Q \geq P$, due to it receiving the token
    first. More precisely, $P = Q$ or $P = Q - 1$. In the first case, P would have just received the message and incremented it's counter. In the second case, vice versa. The question states that P has just sent a message, therefore the starting state must be $P=102$, $Q = 102$.\\

    Following, the two snapshot results that could arise are:

    $snap(P) =  <102, \{m\}>$, $snap(Q) = <103, \{\}>$. In the case where Q sends the message before it receives the marker. Or $snap(P) =  <102, \{\}>, snap(Q) = <103, \{\}>$, when it receives the marker before having sent the message, mandating it to send the marker first. See the excalidraw for a visual solution.
}


\question{}


\begin{enumerate}[(a)]
    \item Yes, to see why - first consider the numbers $\{9,10,11\}$, for each map out the sequence of numbers that follow by applying the function defined in the question (Collatz Conjecture). You can then find that $\{11,34\}$, $\{10,5\}$ and $\{9,28,14\}$ occur as the second, second and third numbers in the sequence respectively. From this, it appears possible that we could simulate the first few steps and check! See excalidraw for the process diagram.
    \item In this case, there are very many possible, different resulting snapshots, depending on the order in which each process decides to prioritize between it's buffers (channels). To see one possible solution, refer to excalidraw.
\end{enumerate}


\end{document}
