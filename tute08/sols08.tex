\documentclass[12pt]{article}
\usepackage{comment}
\usepackage[utf8]{inputenc}
\usepackage{xspace}
\usepackage{gastex}
\usepackage{amsmath}
\usepackage{amssymb}
\usepackage{wrapfig}
\usepackage{tikz}
\usepackage{float}
\usepackage{pgfplots}
\usepackage{booktabs} % For \toprule, \midrule and \bottomrule
\usepackage{pgfplotstable} % Generates table from .csvi
\usepackage{csquotes}
\usepackage{graphicx}
\usepackage{amsmath}
\usepackage{multicol}
\usepackage{tgtermes} % times font
\usepackage[shortlabels]{enumitem}
\usepackage{parskip}

\pagestyle{empty}
\textwidth      165mm
\textheight     252mm
\topmargin      -18mm
\oddsidemargin  -2mm
\evensidemargin -2mm
% \renewcommand{\baselinestretch}{0.96}
\newcommand{\impl}{\mathbin{\Rightarrow}}
\newcommand{\biim}{\mathbin{\Leftrightarrow}}
\newcommand{\id}[1]{\mbox{\textit{#1}}}
\newcommand{\tuple}[1]{\langle #1 \rangle}
\newcommand{\ma}{\mathsf{a}}
\newcommand{\mb}{\mathsf{b}}
\newcommand{\mc}{\mathsf{c}}
\newcommand{\md}{\mathsf{d}}
\newcommand{\nat}{\mathbb{N}}
\newcommand{\intg}{\mathbb{Z}}

\newcounter{question}
\newcommand{\question}[1]{
    \stepcounter{question}
    \thequestion. #1 \hfill
}

\newcommand{\revision}[1]{
    \stepcounter{question}
    \thequestion. #1* \hfill
}



\begin{document}
\topskip0pt
\begin{center}
    {\sc The University of Melbourne
        \\
        School of Computing and Information Systems
        \\
    COMP90020 Distributed Algorithms}
    \bigskip \\
    {\Large\bf Tutorial Week 9: Transactions and Concurrency Control}
    \bigskip \\
\end{center}

\section*{Solutions}

\setcounter{question}{41}

% Serial Equivalence
% Conflicting Operation
% Pessimistic concurrency control: two phase locking, read locking, read-write locks
% Optimistic concurreny control: timestamp ordering, kung robinson
% nested transactions
\question{}


\begin{enumerate}[(a)]
    \item $x_T=read(j,T);y_T=read(i,T);x_U=read(k,U);write(j,44,T);write(i,33,T);commit(T);$ cont- $write(i,55,U);y_U=read(j,U);write(k,66,U);commit(U);$
    \item $x_T=read(j,T);y_T=read(i,T);x_U=read(k,U);write(j,44,T);write(i,33,T);$ cont- $write(i,55,U);commit(T);y_U=read(j,U);write(k,66,U);commit(U);$
    \item $x_T=read(j,T);x_U=read(k,U);write(i,55,U);y_T=read(i,T);y_U=read(j,U);$ cont- $write(k,66,U);commit(U);write(j,44,T);write(i,33,T);commit(T);$
\end{enumerate}

\bigskip

\question{}


\begin{enumerate}[(a)]
    \item There are 24 possilbe interleavings of the nested transactions $T_1,T_2,U_1,U_2$ as every execution is serial (each transaction has a single op). There are 6 possible serially equivalent interleavings for the non-nested case.
    \item Nested transactions allow more serially equivalent interleavings because: i) there is a larger number of serial executions ii) there is more potential for overlap between transactions iii) the scope of the effect of conflicting can be narrowed
\end{enumerate}

\bigskip

\question{}

\begin{enumerate}[(a)]
    \item $a.withdraw(4,T_1);commit(T_1);c.withdraw(3,U_1);commit(U_1);\\b.deposit(4,T_2);commit(T_2);b.deposit(3,U_2);commit(U_2);$
    \item $a.withdraw(4,T_1);commit(T_1);c.withdraw(3,U_1);commit(U_1);\\b.deposit(4,T_2);b.deposit(3,U_2);commit(T_2);commit(U_2);$
    \item  Does not reduce concurrency between siblings T1,T2 or U1,U2 however it does affect families of transactions that have contending writes on the same object.
\end{enumerate}

\question{ (a) and (b) are serially equivalent. (c) and (d) are serially equivalent. Only (b) and (c) could occur under 2PL as (a) and (d) violate the release phase.  }

\question{ Only the last transaction will commit. See excalidraw for more.  }




\end{document}
