\documentclass[12pt]{article}
\usepackage{comment}
\usepackage[utf8]{inputenc}
\usepackage{xspace}
\usepackage{gastex}
\usepackage{amsmath}
\usepackage{amssymb}
\usepackage{wrapfig}
\usepackage{tikz}
\usepackage{float}
\usepackage{pgfplots}
\usepackage{booktabs} % For \toprule, \midrule and \bottomrule
\usepackage{pgfplotstable} % Generates table from .csvi
\usepackage{csquotes}
\usepackage{graphicx}
\usepackage{amsmath}
\usepackage{multicol}
\usepackage{tgtermes} % times font
\usepackage[shortlabels]{enumitem}
\usepackage{parskip}

\pagestyle{empty}
\textwidth      165mm
\textheight     252mm
\topmargin      -18mm
\oddsidemargin  -2mm
\evensidemargin -2mm
% \renewcommand{\baselinestretch}{0.96}
\newcommand{\impl}{\mathbin{\Rightarrow}}
\newcommand{\biim}{\mathbin{\Leftrightarrow}}
\newcommand{\id}[1]{\mbox{\textit{#1}}}
\newcommand{\tuple}[1]{\langle #1 \rangle}
\newcommand{\ma}{\mathsf{a}}
\newcommand{\mb}{\mathsf{b}}
\newcommand{\mc}{\mathsf{c}}
\newcommand{\md}{\mathsf{d}}
\newcommand{\nat}{\mathbb{N}}
\newcommand{\intg}{\mathbb{Z}}

\newcounter{question}
\newcommand{\question}[1]{
    \stepcounter{question}
    \thequestion. #1 \hfill
}

\newcommand{\revision}[1]{
    \stepcounter{question}
    \thequestion. #1* \hfill
}



\begin{document}
\topskip0pt
\begin{center}
    {\sc The University of Melbourne
        \\
        School of Computing and Information Systems
        \\
    COMP90020 Distributed Algorithms}
    \bigskip \\
    {\Large\bf Tutorial Week 9: Transactions and Concurrency Control}
    \bigskip \\
\end{center}

\section*{Notes}


\textbf{Strict executions of transactions:} Generally, it is required that transactions delay both their read and write operations so as to avoid both dirty reads and premature writes. The executions of transactions are called strict if the service delays both read and write operations on an object until all transactions that previously wrote that object have either committed or aborted. The strict execution of transactions enforces the desired property of isolation. [Coulouris pp.690]

\section*{Exercises}

\setcounter{question}{41}

% Serial Equivalence
% Conflicting Operation
% Pessimistic concurrency control: two phase locking, read locking, read-write locks
% Optimistic concurreny control: timestamp ordering, kung robinson
% nested transactions
\question{A server manages the objects $a_1, a_2, ..., a_n$. The server provides two operations for its
clients:}

\begin{itemize}
    \item $read (i)$: returns the $v$ of $a_i$
    \item $write(i, v)$: assigns $v$ to $a_i$
\end{itemize}
The transactions T and U are defined as follows:

\begin{itemize}
    \item $T: x = read(j); y = read (i); write(j, 44); write(i, 33);$
    \item $U: x = read(k); write(i, 55); y = read (j); write(k, 66).$
\end{itemize}


\begin{enumerate}[(a)]
    \item Give serially equivalent interleavings of T and U that are strict
    \item Give serially equivalent interleavings of T and U that are not strict but \textbf{could not produce} cascading aborts
    \item Give serially equivalent interleavings of T and U that \textbf{could produce} cascading aborts
\end{enumerate}

\bigskip

\question{The transfer transactions of T and U are defined as:}

\begin{itemize}
    \item $T: a.withdraw(4); b.deposit(4);$
    \item $U: c.withdraw(3); b.deposit(3);$
\end{itemize}

Suppose that they are structured as a pair of nested transactions

\begin{itemize}
    \item $T_1: a.withdraw(4); T_2: b.deposit(4);$
    \item $U_1: c.withdraw(3); U_2: b.deposit(3);$
\end{itemize}

\begin{enumerate}[(a)]
    \item Compare the number of serially equivalent interleavings of $T_1,T_2,U_1,U_2$ with the number of serially equivalent interleavings of $T$ and $U$.
    \item Explain why the use of these nested transactions generally permits a larger number of serially equivalent interleavings than non-nested ones.
\end{enumerate}

\bigskip

\question{Consider the recovery aspects of the nested transactions defined in the exercise above, assume that a withdraw operation will abort if the account will be overdrawn and that in this case the parent will also abort. }

\begin{enumerate}[(a)]
    \item Describe serially equivalent interleavings of $T_1, T_2, U_1, U_2$ that are strict
    \item Describe serially equivalent interleavings of $T_1, T_2, U_1, U_2$ that are not strict
    \item To what extent does the criterion of strictness reduce the potential concurrency gain of nested transactions?
\end{enumerate}

\question{The transactions T and U are defined as follows:}

\begin{itemize}
    \item $T: x = read(i); write(j, 44);$
    \item $U: write(i,55); write(j, 66);$
\end{itemize}

Initial values $a_i = 10, a_j = 20$. Which of the following interleavings are serially equivalent, and which could occur with two-phase locking?

\begin{center}
    \includegraphics[width=0.75\textwidth]{2pc.png}
\end{center}

\bigskip

\question{Consider three concurrent transactions below}

\begin{center}
    \includegraphics[width=0.75\textwidth]{ghosh.png}
\end{center}

Consider concurrency control by timestamp ordering. Which of these three concurrent transactions will commit?



\end{document}
