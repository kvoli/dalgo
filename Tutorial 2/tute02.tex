
\documentclass[12pt]{article}
\usepackage{comment}
\usepackage[utf8]{inputenc}
\usepackage[backend=biber,style=alphabetic,sorting=ynt]{biblatex}
\addbibresource{refs.bib}
\usepackage{xspace}
\usepackage{gastex}
\usepackage{amsmath}
\usepackage{amssymb}
\usepackage{wrapfig}
\usepackage{tikz}
\usepackage{minted}
\usepackage{float}
\usepackage{pgfplots}
\usepackage{booktabs} % For \toprule, \midrule and \bottomrule
\usepackage{pgfplotstable} % Generates table from .csvi
\usepackage{csquotes}
\usepackage{graphicx}
\usepackage{amsmath}
\usepackage{multicol}
\usepackage{tgtermes} % times font
\usepackage[shortlabels]{enumitem}
\usepackage{parskip}

\pagestyle{empty}
\textwidth      165mm
\textheight     252mm
\topmargin      -18mm
\oddsidemargin  -2mm
\evensidemargin -2mm
% \renewcommand{\baselinestretch}{0.96}
\newcommand{\impl}{\mathbin{\Rightarrow}}
\newcommand{\biim}{\mathbin{\Leftrightarrow}}
\newcommand{\id}[1]{\mbox{\textit{#1}}}
\newcommand{\tuple}[1]{\langle #1 \rangle}
\newcommand{\ma}{\mathsf{a}}
\newcommand{\mb}{\mathsf{b}}
\newcommand{\mc}{\mathsf{c}}
\newcommand{\md}{\mathsf{d}}
\newcommand{\nat}{\mathbb{N}}
\newcommand{\intg}{\mathbb{Z}}
\renewcommand*{\bibfont}{\footnotesize}

\newcounter{question}
\newcommand{\question}[1]{
    \stepcounter{question}
    \thequestion. #1 \hfill
}

\newcommand{\revision}[1]{
    \stepcounter{question}
    \thequestion. #1* \hfill
}



\begin{document}
\topskip0pt
\begin{center}
{\sc The University of Melbourne
\\
School of Computing and Information Systems
\\ 
COMP90020 Distributed Algorithms}
\bigskip \\
{\Large\bf Tutorial Week 3: Logical Clocks}
\bigskip \\
\end{center}
\section*{Notes}
\textit{Clock Condition.} For all events $a,b$: if $a \rightarrow b$ then $C(a) < C(B)$. This is satisfied if the following two conditions hold:
\begin{itemize}
    \item $C1$. If $a$ and $b$ are events in process $p_i$ and $a$ comes before $b$, then $C_i(a) < C_i(b)$
    \item $C2$. If $a$ is the sending of a message by process $P_i$ and $b$ is the receipt of that message by process $P_j$, then $C_i(a) < C_j(b)$
\end{itemize}

\section*{Exercises}

\setcounter{question}{9}

\question{Consider the following sequences of events at processes $p_0, p_1, p_2$ and $p_3$. Here $s_i$ and $r_i$ are corresponding send and receive events for all $i$, while $a$ and $b$ are internal events.}
\[
\begin{array}{llllll}
     p_0: & s_1 & s_2 & r_5  \\
     p_1: & r_2 & s_5 \\
     p_2: & r_1 & a & s_4 & r_3 & r_6 \\
     p_3: & s_3 & r_4 & b & s_6
\end{array}
\]



Use Lamport's logical clock to assign clock values to these events (A diagram may help).

\question{Give an example where $L(a) < L(b)$, while $a$ and $b$ are concurrent events.}

\question{By considering a chain of zero or more messages connecting events $e$ and $e'$ and using induction, show that $e \rightarrow e' \Rightarrow L(e) < L(e')$}

\question{Using the same set of sequences from exercise \textit{10}, draw a send/receive diagram to give Vector timestamps to each of the events.}

\question{Show that $V_j[i] \leq V_i[i]$ for all $i,j$}

\question{Show that $e \rightarrow e' \Rightarrow V[e] \rightarrow V[e']$}

\end{document}