\documentclass[12pt]{article}
\usepackage{comment}
\usepackage[utf8]{inputenc}
\usepackage{xspace}
\usepackage{gastex}
\usepackage{amsmath}
\usepackage{amssymb}
\usepackage{wrapfig}
\usepackage{tikz}
\usepackage{float}
\usepackage{pgfplots}
\usepackage{booktabs} % For \toprule, \midrule and \bottomrule
\usepackage{pgfplotstable} % Generates table from .csvi
\usepackage{csquotes}
\usepackage{graphicx}
\usepackage{amsmath}
\usepackage{multicol}
\usepackage{tgtermes} % times font
\usepackage[shortlabels]{enumitem}
\usepackage{parskip}

\pagestyle{empty}
\textwidth      165mm
\textheight     252mm
\topmargin      -18mm
\oddsidemargin  -2mm
\evensidemargin -2mm
% \renewcommand{\baselinestretch}{0.96}
\newcommand{\impl}{\mathbin{\Rightarrow}}
\newcommand{\biim}{\mathbin{\Leftrightarrow}}
\newcommand{\id}[1]{\mbox{\textit{#1}}}
\newcommand{\tuple}[1]{\langle #1 \rangle}
\newcommand{\ma}{\mathsf{a}}
\newcommand{\mb}{\mathsf{b}}
\newcommand{\mc}{\mathsf{c}}
\newcommand{\md}{\mathsf{d}}
\newcommand{\nat}{\mathbb{N}}
\newcommand{\intg}{\mathbb{Z}}

\newcounter{question}
\newcommand{\question}[1]{
    \stepcounter{question}
    \thequestion. #1 \hfill
}

\newcommand{\revision}[1]{
    \stepcounter{question}
    \thequestion. #1* \hfill
}



\begin{document}
\topskip0pt
\begin{center}
    {\sc The University of Melbourne
        \\
        School of Computing and Information Systems
        \\
    COMP90020 Distributed Algorithms}
    \bigskip \\
    {\Large\bf Tutorial Week 8: Distributed Consensus}
    \bigskip \\
\end{center}

\section*{Notes}


\section*{Solutions}

\setcounter{question}{35}

\question{
    This is a specific case of the coordinated attack where $n = 2$.

    The simple and accepted solution is to show that, given some series of sends and acks(), we can never
    be certain that the last message sent has arrived. Therefore we can remove that message for the sending
    process - since it doesn't change anything for that process $p_{sender}$ unless it subsequently receives and acknowledgement and it shouldn't change the result of agreement. This continues until there are no messages. Clearly
    something isn't right here.

    Another way to consider this in simple terms is that since both generals know that their messages may be lost,
    they will want confirmation of delivery to the other general via a subsequent message. Then, how can the responding general be certain that their acknowledgement has also been acknowledged? They need an acknowledgement of the acknowledgement! .... and hopefully you can see this degrade.

    The longer form follows:

    We can use a indistinguishably proof here to show that coordinated attack is impossible.

    Execution A is indistinguishable form Execution B for some process $p$ if $p$ sees the same things
    in both executions

    If A is indistinguishable from B for p, then b can't tell which of these two possible configurations
    it is in and returns the same output for both.

    Consider some chain of executions $A = A_0,A_1,..A_k = B$, where each $A_i$ is indistinguishable from $A_{i+1}$ for process $p_i$.

    If we are solving agreement, then it must be the case that since pi outputs the same value in $A_i$ and $A_{i+1}$, every
    process must output the same value in $A_i$ and $A_i+1$.

    Induction on $k$, every process outputs the same value in $A$ and $B$, even though $A$ and $B$ may be very different executions.
    Lets use this construction to show that there exists some chain of executions from $A$, where all inputs are $1$ and all messages
    are delivered and arrive at a configuration where all processes have input 0 and no messages are lost.

    let $A_0$ be the starting configuration above. We build $A_1, A_2,... A_k$ by pruning messages. Consider $A_i$ and let $m$ be some message
    that is delivered in the last round in which any message is delivered. Construct $A_{i+1}$ by not delivering $m$. While $A_i$ and $A_{i+1}$
    is distinguishable by the recipient of $m$, on the assumption that $n > 1$ - there is some other process that can't tell whether
    $m$ was delivered or not.

    Likewise construct $A_{k+1}$ through $A_{k+n}$ by changing the input of a process $p_i$ $0 \mapsto 1$.
    Lastly, add back in messages $A_{k+n}$ through $A_{k+n+k'}$ in the reverse order in which they were removed.

    In this case, if agreement holds it must be the case that all processes agree on the same decision in $A_0$ and $A_{k+n+k'}$

    yet if integrity holds, it must be the case that $A_0 = 0$ and $A_k+n+k' = 1$. So either agreement or validity is violated in
    some execution.
}

\question{


    See excalidraw. The key point here is that we need to be specific in our construction when choosing what the lying (byzantine) process says.
}

\question{}

\begin{enumerate}[(a)]
    \item Yes, some process pj broadcasts its message. The other processes then run BA with message they received as input.  \item Yes, each process runs BB as the initializer. Then after n rounds (one for each process), take the majority
\end{enumerate}

\question{ Three members say yes, three members say no. One member doesn't respond at all.  No decision can be reached.
}


\question{Sometimes we can, sometimes we cannot. See the excalidraw for an example. For questions like these, its easy to make it really complicated - instead try and draw out the simplest possible scenarios.}


\end{document}
